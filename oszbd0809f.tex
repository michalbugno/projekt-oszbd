\documentclass[12pt]{article}

\usepackage{geometry}
\usepackage[utf8]{inputenc}
\usepackage[polish]{babel}
\usepackage{polski}
\usepackage{hyperref}
\usepackage{graphicx}
\usepackage{verbatim}
\usepackage{acronym}
\usepackage{fancyhdr}
\usepackage[usenames]{color}

\hypersetup{
  linkbordercolor={1 1 1},
  urlbordercolor={1 1 1},
  colorlinks=true
}

\pagestyle{fancy}
\cfoot{}
\rfoot{\thepage}

\author{\textbf{Michał Bugno} \and \textbf{Antek Piechnik} \and prowadzący: \textbf{dr inż. Robert Marcjan}}
\title{
Analiza oraz wizualizacja danych meteorologicznych dla wybranych ośrodków narciarskich
\begin{figure*}[h]
  \centerline{
    \includegraphics[scale=1.0]{images/logo_agh.pdf}
  }
\end{figure*}
}

\begin{document}

\maketitle
\newpage
\tableofcontents
\newpage

\section{Wstęp}
\subsection{Wizja}
Głównym zadaniem projektu jest poznanie struktury
danych typu GIS (geographical information system), jak również analiza oraz
wykorzystanie tego typu danych w wizualizacji danych meteorologicznych. System
docelowo ma za zadanie przedstawienie sytuacji meteorologicznej na podstawie
danych zbieranych na bieżąco jak również danych historycznych zgromadzonych
poprzednio. System ma również mieć możliwość udostępniania danych/wizualizacji
historycznych na życzenie użytkownika. Do celów badania wydajności systemu
wykorzystywane będą dane z przynajmniej dwóch źródeł informacji
meteorologicznej, podczas gdy system ma domyślnie obsługiwać 4-5 stacji
narciarskich (po kilka punktów na każdą stację).

\subsection{Ocena ryzyka}
Technologia Django (w konsekwencji również Python) daje dobre perspektywy rozwoju:
Python jest językiem bogatym w biblioteki (m.in. do Oracle) i jako język dynamiczny
daje możliwość łatwego rozszerzania aplikacji. Dobrze rokuje także projekt GeoDjango
(\url{http://geodjango.org/}) w związku z czym można sądzić, że nie napotkamy na
większe problemy implementacyjne (związane z technologią).

\section{Ogólna struktura systemu}

\subsection{Baza danych}
Wybraną bazą danych jest Oracle. Wyboru dokonaliśmy głównie ze względu na
możliwość dokładnego poznania tego produktu w ramach projektu jak również ze
względu na obszerne wsparcie dla danych GIS - Oracle Spatial.

\subsection{Technologia}
System jest tworzony w technologii Python Django -- posiada ona wsparcie dla
baz Oracle (w tym Spatial) oraz jest prosta i przejrzysta zapewniając
bardziej elastyczny rozwój.

\subsection{Crawler}
Aplikacja jest w rzeczywistości skryptem mającym na celu pobranie
odpowiednich danych z wcześniej przygotowanych źródeł (stron internetowych
    udostępniających informacje meteorologiczne dla konkretnych ośrodków).
Będzie on miał również możliwość aktualizowania bazy danych o pobrane
informacje, po uprzednich skonwertowaniu ich do odpowiedniego formatu.

\subsection{Wizualizacja}
Wizualizacja zostanie utworzona w oparciu o dane wygenerowane przez kontroler
analizy danych oraz o API systemu Google Maps który pozwoli na estetyczną
wizualizację osiągniętych wyników analizy.

\includegraphics[width=35em]{images/data_flow_diagram.pdf}

\subsection{Entity Relationship Diagram}
Poniższy model danych nie jest dokładny, to znaczy tabele \texttt{resorts}
oraz \texttt{worldborders} są już stworzone natomiast pomiary zostaną dodane
dopiero po stworzeniu crawlera w związku z czym model może nie wyglądać tak samo.

\includegraphics[width=35em]{images/erd_diagram.pdf}

\section{Aplikacja Django}
\subsection{Modele}
Wszystkie modele zawierają pole objects typu \texttt{models.GeoManager} który odpowiada za wspieranie zapytan Spatialowych.
Zdefiniowane w aplikacji modele to:
\begin{description}
\item[WorldBorders] odpowiada za przechowywanie granic państw. Pola modelu:
  \begin{description}
  \item[name] nazwa państwa
  \item[lat] szerokość geograficzna
  \item[lon] długość geograficzna
  \item[mpoly] typ \texttt{SDO\_GEOMETRY} przechowywujący \texttt{MultiPolygon} który
    definiuje granice państwa
  \end{description}
\item[Resorts] przechowuje ośrodki dla których posiadamy dane pogodowe. Pola:
  \begin{description}
    \item[name] nazwa miasta
    \item[position] dwuwymiarowa geometria \texttt{Point} przechowywująca długość i
      szerokość geograficzną miasta
  \end{description}
\item[MeasuresResorts] przechowuje dane dotyczące pomiarów w ośrodkach na różnych wysokościach.
  \begin{description}
    \item[resort] (FK) klucz obcy do tabeli Resorts, przechowuje id resortu którego dotyczy pomiar.
    \item[altitude] wysokośc n.p.m.
  \end{description}
\item[Measures] przechowuje poszczególne pomiary.
  \begin{description}
    \item[measure\_resort] (FK) klucz obcy do tabeli MeasuresResorts, przechowuje id ktorego konkretny pomiar dotyczy
    \item[taken\_at] typ \texttt{Datetime} przechowujący czas pomiaru.
    \item[max\_temp] maksymalna temperatura
    \item[min\_temp] minimalna temperatura
  \end{description}
\end{description}


\subsection{Metody}
\subsubsection{Wybranie obiektu z bazy}
\texttt{austria = WorldBorders.objects.filter(pk=Austria)[0]} \\
\texttt{first\_resort = Resorts.objects.all()[0]}

\subsubsection{Państwo, w którym leży resort}
\texttt{country = first\_resort.country()} \\
\texttt{print country.name \# => ``Austria``}

\subsubsection{Najbliższe resorty}
\texttt{resorts = first\_resort.within\_distance(30)} \\
\texttt{  \# => resorty w odległości 30km od danego (bez danego)}

\subsection{SQL}
\subsubsection{Stworzenie tabel}
Kluczowy dla projektu jest oczywiście typ \texttt{SDO\_GEOMETRY} który umożliwia wykonywanie
specyficznych zapytań geograficznych. Reszta pól to dodatkowe informacje na temat państwa/miasta.
\begin{verbatim}
CREATE TABLE "WORLD_WORLDBORDERS" (
    "NAME" NVARCHAR2(50) NOT NULL PRIMARY KEY,
    "LAT" DOUBLE PRECISION NOT NULL,
    "LON" DOUBLE PRECISION NOT NULL,
    "MPOLY" MDSYS.SDO_GEOMETRY NOT NULL
);
\end{verbatim}

\begin{verbatim}
CREATE TABLE "WORLD_RESORTS" (
    "NAME" NVARCHAR2(50) NOT NULL PRIMARY KEY,
    "POSITION" MDSYS.SDO_GEOMETRY NOT NULL
);
\end{verbatim}

Tabela pomiarów posiada dodatkowo klucz obcy do ośrodków aby pomiar można było zaklasyfikować do
danego ośrodka.
\begin{verbatim}
CREATE TABLE "WORLD_MEASURES" (
    "ID" NUMBER(11) NOT NULL PRIMARY KEY,
    "RESORT_ID" NUMBER(11) NOT NULL REFERENCES "WORLD_RESORTS" ("ID")
         DEFERRABLE INITIALLY DEFERRED,
    "TEMP" NUMBER(11) NOT NULL,
    "TAKEN_AT" DATE NOT NULL
);
\end{verbatim}

\paragraph{Ustawienia metryki}
Aby Oracle wiedział, jak wygląda metryka, należy poinformować go ustalając odpowiednie wartości graniczne
oraz dokładność dla kolumn Spatial. W tym wypadku informujemy, że kolumna \texttt{MPOLY} tabeli
\texttt{WORLD\_WORLDBORDERS} posiada zakres długości od -180 do 180 oraz szerokości od -90 do 90 z dokładnością
co 0.05.

\begin{verbatim}
INSERT INTO USER_SDO_GEOM_METADATA
   ("TABLE_NAME", "COLUMN_NAME", "DIMINFO", "SRID")
   VALUES (
    'world_worldborders',
    'mpoly',
    MDSYS.SDO_DIM_ARRAY(
      MDSYS.SDO_DIM_ELEMENT('LONG', -180.0, 180.0, 0.05),
      MDSYS.SDO_DIM_ELEMENT('LAT', -90.0, 90.0, 0.05)
    ),
    4326
  );
\end{verbatim}

\paragraph{Indeksy}
Aby zapytania mogły funkcjonować należy stworzyć indeksy na kolumnach spatial. Służy do tego celu polecenie
\begin{verbatim}
CREATE INDEX "WORLD_RESORTS_POSITION_ID"
ON "WORLD_RESORTS"("POSITION")
INDEXTYPE IS MDSYS.SPATIAL_INDEX;
\end{verbatim}

\paragraph{Sekwencje}
Warto wspomnieć, że część tabeli ma klucze główne liczbowe i aby nie przejmować się ich numerowaniem
stworzyć należy sekwencje. Do tego celu użyliśmy:
\texttt{CREATE SEQUENCE WORLD\_WORLDBORDERS\_SQ}
i analogiczne dla pozostałych tabel.

\subsubsection{Spatial queries}
\paragraph{Ośrodki w pobliżu danego ośrodka}
\begin{verbatim}
SELECT "WORLD_RESORTS"."ID", "WORLD_RESORTS"."NAME",
       SDO_UTIL.TO_WKTGEOMETRY("WORLD_RESORTS"."POSITION")
FROM "WORLD_RESORTS"
WHERE
    (SDO_WITHIN_DISTANCE("WORLD_RESORTS"."POSITION",
        SDO_GEOMETRY(POINT (10.7498 46.96297), 4326),
        \'distance=20000.0\') = \'TRUE\'
    AND NOT ("WORLD_RESORTS"."ID" = 832))
\end{verbatim}
\texttt{SDO\_WITHIN\_DISTANCE} to funkcja sprawdzająca czy geometria z pierwszego argumentu znajduje się w pewnej
odległości od geometrii drugiego. Jak widać drugą geometrię tworzymy przedstawiając dane geograficzne ośrodka
w postaci Well-Known Text: \texttt{POINT(10.749, 46.962)} pierwsza jest natomiast do tej postaci konwertowana
przez funkcję \texttt{TO\_WKTGEOMETRY}. Trzeci argument to odległość jako liczba metrów. Cała
funkcja zwraca true gdy warunek spełniony.

\paragraph{Państwo, w którym znajduje się ośrodek}
\begin{verbatim}
SELECT "WORLD_WORLDBORDERS"."ID", "WORLD_WORLDBORDERS"."NAME",
       "WORLD_WORLDBORDERS"."LAT", "WORLD_WORLDBORDERS"."LON",
       SDO_UTIL.TO_WKTGEOMETRY("WORLD_WORLDBORDERS"."MPOLY")
FROM "WORLD_WORLDBORDERS"
WHERE SDO_CONTAINS("WORLD_WORLDBORDERS"."MPOLY",
    SDO_GEOMETRY(POINT (10.7498 46.96297), 4326)) = \'TRUE\'
\end{verbatim}
W tym wypadku używamy funkcji \texttt{SDO\_CONTAINS} która zwraca prawdę, gdy druga gemoetria całkowicie zawiera pierwszą.
W naszym przypadku pierwszą geometrią jest wielobok przedstawiający granice państwa w tabeli państw natomiast druga to
punkt reprezentujący ośrodek. W ten sposób zwracamy wszystkie państwa których granice obejmująten punkt (w większości
przypadków będzie to jeden rekord).

\subsection{Integracja z GoogleMaps API}
API GoogleMaps jest w języku JavaScript. Wszystkie funkcje, ktorych używamy są zdefiniowane w widokach znajdujących
się w katalogu \texttt{world/templates}.

Głównym widokiem jest \texttt{layout.html} i on definiuje mapę i podstawowe funkcje wyświetlania resortów. Należy pamiętać,
że GoogleMaps działają tylko dla domeny, która została podana przy generowania klucza, dlatego jeśli domena będzie inna
(aktualnie \texttt{http://localhost:8000}) należy \href{http://code.google.com/apis/maps/signup.html}{wygenerować nowy klucz}
i zmienić widok \texttt{world/templates/layout.html}.

\subsubsection{Granice Austrii}
Granice wyświetlane są przez API KML-owe GoogleMaps. Reprezentację KML granic państwa mozna łatwo wyciągnąć z bazy
danych, jednak API nie pozwala na dynamiczne wyświetlanie plików KML i muszą one być dostępne publicznie w Internecie.
Z tego powodu wproowadzona jest redundancja danych: w pliku \texttt{world/templates/layout.html} znajduje się linia
\begin{verbatim}
var kml = new GGeoXml("http://student.agh.edu.pl/msq/austria.kml");
\end{verbatim}
która definiuje położenie pliku do wyświetlania. W razie potrzeby taki plik można wygenerować za pomocą następujących
poleceń:

\begin{verbatim}
python manage.py shell
> from world.models import WorldBorders
> austria = WorldBorders.objects.filter(pk="Austria")[0]
> print austria.mpoly.kml
\end{verbatim}

\subsection{Fikstury}
Dane do aplikacji mogą zostać ponownie załadowane do bazy (włącznie z uprzednim jej wyczyszczeniem). W plikach
\texttt{worldborders.fixtures} i \texttt{resorts.fixtures} znajdują się odpowiednie dane w prostym formacie tekstowym.
Aby załadować je do bazy należy wykonać następujące polecenia:
\begin{verbatim}
python manage.py shell
> from world import load
> load.load_fixtures()
> # lub load.load_worldborders()
> # lub load.load_resorts()
\end{verbatim}
Metoda \texttt{load.load\_fixtures()} czyści bazę i tworzy wszystkie fikstury natomiast metody \texttt{load.load\_worldborders()}
i \texttt{load.load\_resorts()} działają na poszczególnych tabelach.

\subsection{Wizualizacja danych}
W celu wizualizacji danych ściśle powiązanych z GIS wykorzystaliśmy dostępne w Oracle Spatial metody do pracy na wcześniej utworzonych oraz do tworzenia struktur GIS. Poza tym skorzystaliśmy z modułu PIL (Python Imaging Library) do tworzenia obrazów z otrzymanych danych typu WKT (Well-known text).

\subsection{Metody Oracle Spatial}

\subsubsection{SDO\_UTIL.TO\_WKTGEOMETRY}
Konwertuje zadany obiekt typu SDO\_GEOMETRY do typu WKT - well-known text. Argumentem jest obiekt typu SDOGEOMETRY. 
Well known text (WKT) pozwala nam na sprawna manipulacje otrzymanym polygonem chociazby z poziomu biblioteki wizualizujacej pythona (PIL).
\subsubsection{SDO\_GEOM.SDO\_BUFFER}
Tworzy wokół obiektu strefę buforową o zadanych parametrach (odległość oraz tolerancja). Pierwszym parametrem jest odległość czyli rozmiar zadanego buforu. Bufor może mieć wartość ujemną, tak aby utworzony obiekt w wiekszej skali \'obcinany byl\' z niepotrzebnych kawałków. Drugim parametrem jest tolerancja. Tolerancja służyć ma do korygowania niedoskonałości zarówno w posiadanych danych, jak i rowniez utworzonych zapytań z nim związanych. Stosujemy dwa bufory, tak aby jeden z nich korygował niedoskonałości wcześniej zastosowanego.
\subsubsection{SDO\_GEOM.SDO\_CONVEXHULL}
Argumentem jest obiekt SDO\_GEOMETRY, metoda zwraca obiekt SDO\_GEOMETRY będący wypukłą otoczką (convex hull) obiektu. Parametrem drugim jest tolerancja (opisana punkt wyżej).
\subsubsection{MDSYS.SDO\_GEOMETRY}
Tworzy strukturę geometryczną o zadanych parametrach. Dokładny opis metody jest mocno zawiły, przyjmuje oa jako argumenty punkty mające stanowić polygon jak i również typ polygonu do utworzenia (prostokat, romb, wielobok nieregularny)
\subsubsection{SDO\_GEOM.SDO\_ELEM\_INFO\_ARRAY}
Informuje o formacie danych podawanych przy tworzeniu polygonu (pierwszym argumentem jest numer poczatkowy z mapy punktow, drugi argument definiuje typ podawanych elementow (np. relacje jak w wypuklym polygonie)).
\subsubsection{SDO\_ORDINATE\_ARRAY}
Formatuje zadane punkty (oddzielone kolejno przecinkami) do formatu akceptowanego przez powyższe funkcje.
\subsection{Konkretne zadania wizualizujące}
\subsubsection{Wizualizacja izoterm}
Do wizualizacji izoterm wykorzystaliśmy zapytania GeoDjango pozwalające na wybranie Resortów w określonej odległości od zadanego Resortu początkowego. Znajduje sie ono pod urlem iso.

\begin{verbatim}
points_data = Resorts.objects.filter(
              position__dwithin=(self.position, distance_km),
              measuresresorts__measures__min_temp__lt=temperature
              ).unionagg().coords
\end{verbatim}

Następnie po przekonwertowaniu danych do odpowiedniego formatu korzystamy z nich przy generowaniu zapytania SQL.

\begin{verbatim}
SELECT SDO_UTIL.TO_WKTGEOMETRY(
 SDO_GEOM.SDO_BUFFER(
  SDO_GEOM.SDO_BUFFER(
   SDO_GEOM.SDO_CONVEXHULL(
    MDSYS.SDO_GEOMETRY(
     2003,
     4326,
     NULL,
     SDO_ELEM_INFO_ARRAY(1, 2003, 1),
     SDO_ORDINATE_ARRAY(%s)),
    50),
   100, 50, 'unit=m'),
  -20, 50, 'unit=m')
 ) FROM WORLD_WORLDBORDERS;" % points_str
\end{verbatim}

Zaczynając od wewnątrz, metoda SDO\_GEOMETRY o parametrach 2003 (polygon wypukly), 4326 (system koordynacji, do dostosowywania polygonu do reszty danych GIS, pierwszy bufor o tolerancji 50m oraz dystansie 100m wokol punktow, drugi bufor ucina (buforuje 'do wewnatrz') o 20m.  Całośc zamieniana jest do formatu well-known text.

Zwracane dane to (w przypadku pomyślnego wyszukania odpowiednich danych resortów, POLYGON w formacie WKT (well-known text). Następnie polygon przetwarzany jest przez odpowiednie metody w PIL (python imaging library) i wraz z danymi GIS danego kraju tworzony jest obraz przedstawiający izotermę następującej postaci:

\includegraphics[width=35em]{images/isotherm.png}

\section{Wyniki wizualizacji isoterm.}

W celu dopracowania systemu wizualizacji zaczęliśmy grupować zestawy ośrodków w zadanej odległości od głównego o zbliżonych pomiarach temperatur. Wprowadziliśmy różną kolorystykę w celu usprawnienia odczytu wykresów, jak również wprowadziliśmy opis legendy (która również generowana jest optymalnie do dostępnych wskazań temperatury w zadanym dniu. Mimo nie do końca kompletnych i sprawnych danych udało nam się osiągnąć zadowalające wyniki:

\includegraphics[width=35em]{images/iso_Lachtal_2008-03-02.png}

\includegraphics[width=35em]{images/iso_Pitztal_2008-03-02.png}

\includegraphics[width=35em]{images/iso_Kitzbuhel_2008-03-02-1.png}

\newpage

\section{Crawler}
Crawler jest napisany w języku Ruby i służy do pobierania danych ze strony
\\\url{http://www.snow-forecast.com}.
Docelowo będzie to prosty skrypt oparty o metodologię \emph{Extract--Transform--Load}.
\subsection{Extract}
\begin{itemize}
\item uruchamiamy skrypt z parametrem adresu strony (w zasadzie chodzi o wybrany szczyt)
\item skrypt analizuje stronę za pomocą parsera HTML+XML Nokogiri
\item dane zapisywane są w prostej postaci w tablicy
\item skrypt znajduje link do danych z poprzedniego okresu, odwiedza go i powtarza proces
\end{itemize}
Dane zapisywane są przez
\begin{verbatim}
file = File.open(filename, "w")
file.write(Marshal.dump(data))
\end{verbatim}
Aby je odczytać wystarczy
\begin{verbatim}
data = Marshal.load(File.read(filename))
\end{verbatim}

\section{Raport prac}
Plan jest ułożony malejąco względem priorytetów.
\subsection{Crawler}
\begin{itemize}
  \item część \emph{Transform} przerabiające surowe dane na potrzeby aplikacji
  \item część \emph{Load} tworząca SQL-e i wykonująca je w kontekście bazy
\end{itemize}

\subsection{Import danych}
\begin{itemize}
  \item utworzenie odpowiednich tabel do przetrzymywania danych, utworzenie relacji między tabelami
  \item zmapowanie tabel w modelach Django
  \item ustalenie formatu danych oraz ich konwersja do formatu odpowiadającego modelowi w bazie danych
  \item zaimportowanie danych do bazy danych
\end{itemize}

\subsection{Analiza danych}
\begin{itemize}
  \item umożliwienie wyświetlania prognozowanych danych dla resortów, dla których istnieją pobrane dane
        (podpierając się GIS)
  \item listowanie resortów w zadanej odległości
  \item możliwość zacieśnienia przeszukiwanych/porównywanych resortów tylko i wyłącznie do określonego regionu
\end{itemize}

\subsection{Wizualizacja danych}
\begin{itemize}
  \item generowanie wykresów na podstawie danych historycznych (np. temperatury z ostatniego tygodnia)
  \item wizualizacja izoterm na mapie Austrii, bazujacych na dostępnych danych, o różnych poziomach przedziałów temperatur.
  \item wizualizacja izobar (j.w.)
  \item wizualizacja danych prognozowanych za pomocą Google Maps API
\end{itemize}

\subsection{Dodatkowe}
\begin{itemize}
  \item implementacja wyszukiwania ośrodków
  \item próba wykorzystania Google Maps API do prezentowania dokładniejszej odległości danego ośrodka od
        użytkownika (na podstawie jego aktualnej pozycji np. w preferencjach)
\end{itemize}

\section{Linki}
\begin{description}
  \item[Źródła (repozytorium Git)] \url{http://github.com/michalbugno/projekt-oszbd/}
  \item[Projekt Django] \url{http://www.djangoproject.com/}
  \item[Projekt GeoDjango] \url{http://geodjango.org/}
  \item[API GoogleMaps] \url{http://code.google.com/apis/maps/documentation/}
  \item[System kontroli wersji Git] \url{http://git-scm.com/}
  \item[Python] \url{http://www.python.org/}
  \item[Ruby] \url{http://www.ruby-lang.org/}
  \item[Oracle Spatial] \url{http://download.oracle.com/docs/cd/B10501_01/appdev.920/a96630/toc.htm}
\end{description}

\end{document}

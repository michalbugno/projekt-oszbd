\documentclass[12pt]{article}

\usepackage{geometry}
\usepackage[utf8]{inputenc}
\usepackage[polish]{babel}
\usepackage{polski}
\usepackage{hyperref}
\usepackage{graphicx}
\usepackage{verbatim}
\usepackage{acronym}
\usepackage{fancyhdr}
\usepackage[usenames]{color}

\hypersetup{
  linkbordercolor={1 1 1},
  urlbordercolor={1 1 1},
  colorlinks=false
}

\pagestyle{fancy}
\cfoot{}
\rfoot{\thepage}

\author{Michał Bugno \and Antek Piechnik}
\title{Harmonogram prac}
% W oparciu o bazę danych Oracle z wykorzystaniem technologii Oracle Spatial.

\begin{document}
\maketitle

\section{Dotychczasowy postęp}
\begin{itemize}
\item Utworzono Crawler do danych pogodowych w języku Ruby.
\item Zebrano dane z parunastu zeszlych miesięcy dla ponad 200 kurortów narciarskich zlokalizowanych na terenie Austrii.
\item Stworzono szkielet aplikacji webowej we frameworku Merb. 
Skonfigurowano go oraz połączono z bazą danych Oracle umieszczoną na uczelnianym serwerze.
\item Utworzono oraz zrealizowano w bazie danych podstawowy schemat tabel oraz relacji konieczny do dalszej pracy aplikacji.
\item Wykorzystano OCI8 (ruby-oci8) jako interfejs bazy Oracle dla języka Ruby stanowiącego bazę aplikacji webowej.
\item Przygotowano zestaw danych typu GIS opisujacych cały teren kraju Austriackiego oraz dodatkowe dane zawierającego informacje o elewacji terenu.
\end{itemize}

\section{TODO}
\begin{itemize}
\item Import danych do bazy.
\item Integracja z API GoogleMaps (rozwiazanie lepsze niż korzystanie z wbudowanych narzędzi Oracla)
\item Utworzenie silnika pobierania oraz analizy danych z bazy.
\item Przygotowanie systemu wizualizujacego rezultaty pracy silnika analizy danych.
\end{itemize}

\section{Problemy}
\begin{itemize}
\item Interfejs OCI8 nie obsługuje SDO\_GEOMETRY - 
być może trzeba będzie probować powiązać dane GISowe bezposrednio z GoogleMaps i z tamtego poziomu na nich operowac. 
Alternatywa jest odszukanie niestabilnych rozwiązań obsługi SDO\_GEOMETRY w trunkach ruby-oci8 (poniewaz narazie nie zalezy nam az tak bardzo na stabilnosci całego systemu).
\item Odnalezienie dokładniejszych danych typu GIS dla interesujacych nas obszarów - 
Narazie posiadamy dane GIS opisujace cały kraj oraz elewacje jego terenu. Ciężko jest odnaleźć dużo dokładniejsze dane opisujące konkretne kurorty narciarskie.
\end{itemize}

\end{document}
